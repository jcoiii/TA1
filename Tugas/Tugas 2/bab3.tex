\chapter{\uppercase{Perancangan Sistem}}
{\color{red}\itshape\lipsum[1]}

\section{Spesifikasi Sistem/Disain}
{\color{red}\itshape\lipsum[2]}
\begin{table}[H]
	\caption{Contoh kedua}
	\label{tab:random1}
	\begin{center}  
		\setlength\extrarowheight{5pt}	
		\begin{tabular}{|c|c|c|c|}	
			\hline
			\textbf{$Title_1$} & \textbf{$title_{\mathrm{DUA}}$} & \textbf{Title 3 } & \textbf{Title 4}  \\
			\hline
			1647 &  1.97  &  0.68 &  1.90 \\
			2301 &  2.92  &  1.06 &  2.75 \\
			2969 &  3.23  &  1.16 &  3.78 \\
			4625 &  6.72  &  1.87 &  5.59 \\
			\hline
		\end{tabular}
	\end{center}
\end{table}

{\color{red}\itshape\lipsum[3]}
\section{Rincian Disain}
\subsection{Komponen Utama}
Pada bagian ini, Anda tampilkan disain/sistem utama Anda. Tampilkan rancangan bagian/komponen yang penting saja. Bagian/ skema lebih rinci dapat Anda letakkan pada bagian Lampiran. 

\subsection{Komponen lebih rinci}
Bagan atau skema disain yang berukuran lebih dari ukuran kertas A4 dapat dicetak menggunakan kertas berukuran lebih besar dari A4. Pada saat dikumpulkan, skema ini harus dilipat menjadi berukuran A4. Perhatikan juga bahwa ada ukuran \textit{margin} kiri untuk dokumen ini, sebesar 4 cm. Letakkan gambar skema ini pada bagian Lampiran.\\ 
{\color{red}\itshape\lipsum[4]}

\section{Rencana Pembuatan}
Dalam bagian ini dijabarkan prosedur/rencana kerja/urutan pembuatan dari sistem/disain yang diusulkan. Tampilkan pula jadwal rencana pembuatan disain Anda, menggunakan diagram Gantt Chart. Tambahkan beberapa titik \textit{capstone} (titik capaian) pada rencana Anda. \textit{Capstone} ini penting saat Anda mulai mengerjakan disain, saat Tugas Akhir II nantinya. Hal ini akan membiasakan Anda untuk memenuhi target capaian tertentu pada suatu proses.\\

Bila Gantt Chart Anda terlalu panjang, maka Anda dapat meletakkan Gantt Chart ini pada bagian Lampiran.\\

\begin{figure}
	\centering 
	\includegraphics[width=\textwidth]{gambar/gantt1}
	\caption{Contoh Gantt Chart.}
\end{figure}   
%{\color{red}\itshape\lipsum[5]}

\section{Rencana Pengujian Sistem}
Pada bagian ini, dijabarkan tentang bagaimana Anda akan menguji performa dari disain Anda. Karena hasil disain/simulasi/produk Anda merupakan jawaban dari sebuah masalah, maka perhatikan variabel/ parameter disain awal yang ada pada masalah awal. Contoh dari hal ini antara lain: ukuran/dimensi alat yang dihasilkan, kecepatan perhitungan (pada kasus simulasi), ketepatan mencapai tujuan (akurasi) dan lain sebagainya.
\section{Rincian Biaya}
Jika hasil akhir Tugas Akhir Anda berupa sebuah produk yang memerlu-kan biaya pembuatan atau pembelian material, maka Anda dapat meletakkan tabel kebutuhan biaya pada bagian ini. \prodis{} \tekm{} UNPAR tidak menanggung biaya pembelian material dan proses pemesinan dari disain Anda. Oleh karena itu, sesuaikan disain Anda dengan kemampuan finansial Anda. Jika disain Anda memang '\textit{marketable}', maka disarankan untuk mencari '\textit{investor} untuk proyek Anda. Walau begitu, Program Studi tetap dapat menyediakan/ mengadakan \textbf{sebagian} komponen yang diperlukan, sepanjang komponen tersebut memang telah dianggarkan dalam Rencana Anggaran Program Studi.\\

Hak cipta dan hasil produk Tugas Akhir sepenuhnya menjadi milik \prodis{} \tekm{} \unpar.  \\

{\color{red}\itshape\lipsum[4]}


