%\chapter{ABSTRAK}
\clearpage
\phantomsection
\addcontentsline{toc}{chapter}{Abstrak}
\begin{center}
\textbf{\large  Abstrak}\\[3em]
\end{center}

Abstrak memuat secara komprehensif (singkat, padat dan jelas) masalah/topik yang diangkat dalam penelitian Tugas Akhir serta usulan disain/solusi yang diajukan untuk menyelesaikan masalah/topik tersebut. Abstrak menjelaskan masalah yang diteliti, latar belakangnya, metode penyelesaian yang digunakan, perancangan alat atau sistem yang diusulkan serta kajian terhadap rancangan atau sistem mekatronika yang telah dibuat. Abstrak juga harus dengan jelas menyatakan dan memuat sumbangan/kontribusi hasil Tugas Akhir terhadap salah satu pihak (namun tidak terbatas): masyarakat, jurusan \tekm{} UNPAR, entitas usaha/bisnis tertentu dan/atau ilmu pengetahuan. Semua hal ini harus dapat disampaikan dalam 1 halaman (maksimal 300 kata).

Template ini dibuat oleh Dr. Ir. Ali Sadiyoko, M.T. untuk membakukan format penulisan \lta{} di \prodis{} \tekm, \fti, \unpar.  
\vspace{5mm}

\noindent 
\rule{\textwidth}{0.4pt}
\textbf{Kata kunci:}\\
\emph{\keywords}
