%  Format/Template Draft Laporan Tugas Akhir 
%  di Program Studi Sarjana Teknik Mekatronika, 
%  Fakultas Teknologi Industri, 
%  Universitas Katolik Parahyangan
%  Jl. Ciumbuleuit no 94, Bandung 40141, INDONESIA.
%  Versi : 1.0.0
%  Tanggal : 16 Februari 2019
%  Pembuat: Ali Sadiyoko, Teknik Mekatronika UNPAR.
%  
%%%%%%%%%%%%%%%%%%%%%%%%%%%%%%%%%%%%%%%%%%%%%%%%%%%%%%%%%%%%

\documentclass[12pt,oneside]{book} % Deklarasi kelas dokumen

%Isi identitas Anda dan informasi tentang Tugas Akhir Anda di file ini......
%%%%%%%%%%%%%%%%%%%%%%%%%%%%%%%%%%%%%

% JUDUL TUGAS AKHIR: 
\providecommand{\judul}{Sistem Pendulum Terbalik} % isikan dengan judul TA anda.
\providecommand{\tahun}{2019} % isikan dengan tahun pembuatan laporan TA anda.

% Nama Identitas Mahasiswa:
\providecommand{\namamhsw}{Jonathan Chandra} % isikan dengan nama anda.
\providecommand{\npmmhsw}{2015630028} % isikan dengan NPM anda.

%Isi dengan nama Dosen Pembimbing TA
\providecommand{\pembimbingsatu}{Dr. Ali Sadiyoko, M.T.} 
\providecommand{\pembimbingdua}{Tua Agustinus Tamba, Ph.D} 

%Untuk kata kunci yang ada di abstrak, di sini:
\providecommand{\keywords}{otomatis, lini produksi, transportasi}

%Untuk keywords yang ada di Abstract, di sini:
\providecommand{\keywordsen}{automated, production lane, transportation}  % File identitas Tugas Akhir
% Version 1.0.0. (16 Feb 2019)
% ==================================

% Format kertas
\usepackage{geometry}
\geometry{a4paper, top=3cm,bottom=3cm,left=4cm,right=3cm}
%\usepackage{tgtermes} % Times New Roman
%\usepackage{times} % LATEX Roman

% Package yang diperlukan
\usepackage[colorlinks=true, linkcolor =black, anchorcolor = black, citecolor = black, urlcolor = black]{hyperref}
\usepackage[htt]{hyphenat}
\usepackage{listing}	  % Package untuk listing program	
\usepackage{graphicx}     % Package untuk grafik/ gambar
\usepackage{tabularx}     % Package untuk tabel
\usepackage{float}
\usepackage[hang,nooneline,scriptsize,md]{subfigure}
\usepackage{epsfig}	% Package untuk grafik/ gambar *.eps
\usepackage[font=small,labelfont=bf, labelsep=period]{caption}
\usepackage[onehalfspacing]{setspace}      % Package untuk spasi 1.5
\usepackage{indentfirst}
%\setlength{\parindent}{0.6cm}
\usepackage{parskip}
\usepackage[titletoc]{appendix}
\usepackage{cite}          % Package untuk sitasi menggunakan BibTex
\usepackage[indonesian]{babel} % Package untuk Bahasa Indonesia
\usepackage{array,ragged2e}
\usepackage[utf8]{inputenc}
\usepackage{textcase}
\usepackage{bigstrut}
%\usepackage[T1]{fontenc}  Optional, untuk beberapa simbol khusus
\usepackage{rotating}
%\usepackage{relsize}
\usepackage{textcomp} 	% nice greek alphabet
\usepackage{booktabs}
\usepackage{amssymb,amsthm}
\usepackage{amsmath}
\usepackage{tikz} 
\usetikzlibrary{shapes, arrows,calc,bending,matrix}
\usepackage[american, siunitx]{circuitikz}
\usepackage{lipsum}
%\usepackage{graphicx}
\usepackage{bm}
\usepackage{nicefrac}
\usepackage{pdfpages}
%\usepackage[colorlinks=true, linkcolor =black, anchorcolor = black, citecolor = black, urlcolor = black]{hyperref}
\usepackage{courier}
\usepackage{listings}
\lstset{basicstyle=\small\ttfamily,breaklines=true}
\lstset{framextopmargin=50pt}
\usepackage{enumitem}
\usepackage{titlesec}

%%% Pengaturan global
\graphicspath{{gambar/}}   % letak direktori penyimpanan gambar

\pagestyle{plain}


% Format Judul Bab

\renewcommand{\chaptertitlename}{BAB} 
\titleformat{\chapter}[display]
{\bfseries\Large}
{%
	% \vskip-5em
	\filcenter
	\Large\chaptertitlename \ \thechapter} % --> BAB 1
{0mm}
{\filcenter}
\titleformat*{\section}{\bfseries\large}
\titleformat*{\subsection}{\bfseries}
\titlespacing*{\chapter}{0pt}{0pt}{15mm}
% \titlespacing{command}{left spacing}{before spacing}{after spacing}[right]
% Bagian untuk menghilangkan titik2 di daftar isi, gambar, tabel
\usepackage[subfigure]{tocloft}
\renewcommand{\cftpartleader}{\hfill}
\renewcommand{\cftpartafterpnum}{\cftparfillskip}
\renewcommand{\cftchapleader}{\hfill} 
\renewcommand{\cftsecleader}{\hfill}

%tambah "BAB" x di daftar isi 
%\renewcommand*\cftchappresnum{BAB~}
%\renewcommand\chaptername{BAB}
%\settowidth{\cftchapnumwidth}{\cftchappresnum}
%\renewcommand{\cftchapaftersnumb}{\quad}
%\addtocontents{toc}{
%	%\renewcommand\protect*\protect\cftchappresnum{\MakeUppercase{\chaptername}~}}
%	\protect\renewcommand*\protect\cftchappresnum{\chaptername~}}

%tambah "BAB" x di daftar isi 
\renewcommand*\cftchappresnum{BAB~}
\renewcommand\chaptername{BAB}
\settowidth{\cftchapnumwidth}{\cftchappresnum}
\renewcommand{\cftchapaftersnumb}{\quad}
\addtocontents{toc}{
	%\renewcommand\protect*\protect\cftchappresnum{\MakeUppercase{\chaptername}~}}
	\protect\renewcommand*\protect\cftchappresnum{\chaptername~}}

% Penambahan kata "Gambar" dan "Tabel" di daftar gambar dan daftar tabel
\usepackage{chngcntr}  
%\counterwithout{figure}{chapter}
\renewcommand{\cftfigpresnum}{\bfseries Gambar\ }
\renewcommand{\cfttabpresnum}{\bfseries Tabel\ }
\newlength{\mylenf}
% Jarak jeda antara nomer gambar/ tabel dengan keterangannya.
\settowidth{\mylenf}{\cftfigpresnum}
\setlength{\cftfignumwidth}{\dimexpr\mylenf+3em}
\settowidth{\mylenf}{\cfttabpresnum}
\setlength{\cfttabnumwidth}{\dimexpr\mylenf+3em}
\makeatletter

% listoftable tanpa judul
%\counterwithout{table}{chapter}
\newcommand\daftartabel{%
	\phantomsection
	\@starttoc{lot}}
\makeatother


% Menghilangkan judul bibliografi 
\makeatletter
\renewenvironment{thebibliography}[1]{%
	%     \section*{\refname}%
	%      \@mkboth{\MakeUppercase\refname}{\MakeUppercase\refname}%
	\list{\@biblabel{\@arabic\c@enumiv}}%
	{\settowidth\labelwidth{\@biblabel{#1}}%
		\leftmargin\labelwidth
		\advance\leftmargin\labelsep
		\@openbib@code
		\usecounter{enumiv}%
		\let\p@enumiv\@empty
		\renewcommand\theenumiv{\@arabic\c@enumiv}}%
	\sloppy
	\clubpenalty4000
	\@clubpenalty \clubpenalty
	\widowpenalty4000%
	\sfcode`\.\@m}
{\def\@noitemerr
	{\@latex@warning{Empty `thebibliography' environment}}%
	\endlist}
\makeatother


%lampiran
\makeatletter
\newcommand\appendix@chapter[1]{%
	\refstepcounter{chapter}%
	\orig@chapter*{Lampiran \@Alph\c@chapter \\ #1}\vspace{1em} % Format di lembar lampiran 
%	\addcontentsline{toc}{chapter}{LAMP. \@Alph\c@chapter: #1}%
	\addcontentsline{toc}{chapter}{{} \@Alph\c@chapter. #1} % Format di daftar isi
}

\let\orig@chapter\chapter
\g@addto@macro\appendix{\let\chapter\appendix@chapter}
\makeatother

%%% Tambahan format untuk tabel : cell alignment
\newcolumntype{C}[1]{>{\centering\arraybackslash}m{#1}}
\newcolumntype{R}[1]{>{\raggedleft\arraybackslash}m{#1}}
\newcolumntype{L}[1]{>{\raggedright\arraybackslash}m{#1}}
     % File berisi format Laporan Tugas Akhir

% Daftar beberapa istilah khusus Teknik Mekatronika UNPAR
\newcommand{\prodis}{Program Studi Sarjana}
\newcommand{\tekm}{Teknik Elektro (Konsentrasi Mekatronika)}
\newcommand{\fti}{Fakultas Teknologi Industri}
\newcommand{\unpar}{Universitas Katolik Parahyangan}
\newcommand{\lta}{Proposal Tugas Akhir}
\newcommand{\TA}{Tugas Akhir}
\newcommand{\PTA}{Proposal Tugas Akhir}
\newcommand{\ieee}{\textit{Institute of Electrical and Electronics Engineers}}


%%%% Identitas file PDF
\pdfinfo{
	/Author (\namamhsw)
	/Title  (\judul)
	/CreationDate (\today)
	/Subject (\lta{} Teknik Mekatronika \unpar, INDONESIA)
	/Keywords (\keywords)
}

\begin{document}
\frontmatter	
\begin{titlepage}
\begin{center}

%bagian atas dari halaman
\textbf{\Large{\MakeUppercase{\judul}}}\\[2cm]
\textbf{\large \MakeUppercase \lta}\\[1cm]
Diajukan untuk memenuhi salah satu syarat mengikuti Sidang \lta{} pada mata kuliah Tugas Akhir I (IME 184400-02)\\[1cm]



\large Disusun Oleh: \\
\vspace{5mm}
\textbf{\namamhsw} \\
(\npmmhsw)
\vspace{10mm}

%Logo UNPAR
\includegraphics[width=4cm]{logounpar}\\[1cm]   


\vfill

% bagian bawah
\small
\textbf{\MakeUppercase \prodis}\\
\textbf{\MakeUppercase \tekm}\\
\textbf{\MakeUppercase \fti}\\
\textbf{\MakeUppercase \unpar}\\
\textbf{BANDUNG\\ \tahun}
\end{center}
\end{titlepage}
      % Lembar sampul utama
\thispagestyle{empty}
\begin{center}
%bagian atas dari halaman
\begin{doublespace}
\textbf{\Large{\MakeUppercase{\judul}}}\\[2.5cm]
\end{doublespace}

\textbf{\large \MakeUppercase \lta}\\

Diajukan untuk memenuhi salah satu syarat mengikuti Sidang Proposal pada mata kuliah Tugas Akhir I (IME 184400-02)\\[2cm]

\large Disusun Oleh: \\
\vspace{5mm}
\textbf{\namamhsw} \\
(\npmmhsw)
\vfill

% Isikan dengan data Pembimbing TA Anda.
Dosen Pembimbing : \\
\large{\pembimbingsatu \\ \pembimbingdua} \\


\vfill

% bagian bawah
\textbf{\prodis}\\
\textbf{\tekm}\\
\textbf{\fti}\\
\textbf{\unpar}\\
\textbf{\large \tahun}\\
\end{center}

 % Lembar sampul dalam
% Lembar Persetujuan Pembimbing
\newpage
\thispagestyle{empty}
\begin{center}

\textbf{\large {PERSETUJUAN PEMBIMBING}}\\
\vspace{2cm}
\large Judul \lta :\\[5mm] 
\nohyphens{\textbf{\large \MakeUppercase{\judul}}}

\vspace{10mm}
\large Peserta Tugas Akhir I: \\
\vspace{5mm}

\textbf{\namamhsw} \\
( \npmmhsw{} )
\end{center}
\vspace{5mm}
\justify
{\singlespacing \normalsize Mahasiswa \prodis{} \tekm, \fti, \unpar{} di atas telah melaksanakan proses bimbingan pembuatan Proposal Tugas Akhir dan menyusun \lta{} sesuai ketentuan yang telah ditetapkan serta dinyatakan layak untuk mengikuti Sidang Proposa.}\\

\vspace{5mm}
Bandung,\ldots\ldots\ldots\\

\begin{center}
\textbf{Dosen Pembimbing:}\\
\vspace{30mm}
%\begin{tabular}[l]{L{ 0.4\textwidth}L{ 0.15\textwidth}L{ 0.4\textwidth}}
\begin{tabular}{ccc}
%	{} & {} & {} \\
	\underline{\pembimbingsatu}  & \hspace{3cm} & \underline{\pembimbingdua} \\[0mm]
	Pembimbing I  & \hspace{3cm}  & Pembimbing II\\[0mm]
\end{tabular} 
\end{center}

\hfill


         % Lembar Persetujuan Pembimbing
% Lembar Pernyataan tidak mencontek atau melakukan tindakan plagiat
\thispagestyle{empty}
\begin{center}

\textbf{\large {PERNYATAAN TIDAK MENCONTEK ATAU\\
		MELAKUKAN TINDAKAN PLAGIAT}}\\
\vspace{2cm}

Saya yang bertandatangan dibawah ini, \\
\vspace{5mm}
\textbf{\MakeUppercase{\namamhsw}}
\vspace{5mm}

\singlespacing \normalsize Dengan ini menyatakan bahwa \lta{} dengan judul:\\
\vspace{5mm}
"{\MakeUppercase{\judul}}"\\
\vspace{5mm}
adalah hasil pekerjaan Saya. Seluruh ide, pendapat atau materi dari sumber
lain telah dikutip dengan cara penulisan referensi yang sesuai.

\vspace{5mm}
Pernyataan ini Saya buat dengan sebenar-benarnya dan jika pernyataan ini tidak
sesuai dengan kenyataan maka Saya bersedia menanggung sanksi yang akan dikenakan kepada Saya.\\

\vspace{10mm}
Bandung,\ldots\ldots\ldots\\[20mm]
% Isi bagian \ldots\ldots\ldots dengan tanggal selesai/ penyerahan Laporan TA

\begin{tabular}[c]{C{ 0.3\textwidth}C{ 0.3\textwidth}C{ 0.3\textwidth}}
	{} & {}\\ 
	 & \underline{\textbf{\namamhsw}} &  \\[0mm]
	&  \small NPM: \npmmhsw &  \\[0mm] 
\end{tabular} 


\end{center}
\hfill


    % Pernyataan tidak mencontek/ melakukan tindakan plagiat

\pagenumbering{roman} 
%\chapter{ABSTRAK}
\clearpage
\phantomsection
\addcontentsline{toc}{chapter}{Abstrak}
\begin{center}
\textbf{\large  Abstrak}\\[3em]
\end{center}

Abstrak memuat secara komprehensif (singkat, padat dan jelas) masalah/topik yang diangkat dalam penelitian Tugas Akhir serta usulan disain/solusi yang diajukan untuk menyelesaikan masalah/topik tersebut. Abstrak menjelaskan masalah yang diteliti, latar belakangnya, metode penyelesaian yang digunakan, perancangan alat atau sistem yang diusulkan serta kajian terhadap rancangan atau sistem mekatronika yang telah dibuat. Abstrak juga harus dengan jelas menyatakan dan memuat sumbangan/kontribusi hasil Tugas Akhir terhadap salah satu pihak (namun tidak terbatas): masyarakat, jurusan \tekm{} UNPAR, entitas usaha/bisnis tertentu dan/atau ilmu pengetahuan. Semua hal ini harus dapat disampaikan dalam 1 halaman (maksimal 300 kata).

Template ini dibuat oleh Dr. Ir. Ali Sadiyoko, M.T. untuk membakukan format penulisan \lta{} di \prodis{} \tekm, \fti, \unpar.  
\vspace{5mm}

\noindent 
\rule{\textwidth}{0.4pt}
\textbf{Kata kunci:}\\
\emph{\keywords}
     % File Abstrak
\clearpage
\phantomsection
\addcontentsline{toc}{chapter}{Kata Pengantar}
\begin{center}
 \textbf{\large Kata Pengantar}\\[3em]
\end{center}
%-----------------------------------------

Kata Pengantar adalah salah satu bagian pada sebuah dokumen (buku, laporan atau dokumen lain) yang berisi ungkapan rasa syukur penulis, ucapan-ucapan terima kasih, tujuan dan harapan penulis terhadap karyanya serta kritik atau saran yang membangun. Contoh bagian awal Kata Pengantar adalah sebagai berikut:\\
\begin{quote}\color{red}
Puji syukur penulis panjatkan ke hadirat Allah SWT, karena dengan rahmat-Nya lah penyusunan \lta{} ini dapat diselesaikan dengan baik. \lta{} yang berjudul "\judul"  disusun, sebagai syarat untuk mengikuti Sidang Proposal pada mata kuliah Tugas Akhir I (IME 184400-02) pada \prodis{} \tekm{} \unpar.
\end{quote}

 Kata pengantar terdiri atas tiga bagian utama yaitu pembukaan, isi, dan penutup, yaitu:

\begin{enumerate}
	\item Pembukaan kata pengantar biasanya berisi ucapan rasa syukur penulis atas selesainya penulisan. Diawali dengan rasa syukur dan pujian kepada Tuhan Yang Maha Esa, dilanjutkan kepada tim pembimbing, keluarga, sahabat, dan orang-orang yang membantu dalam proses penyusunan karya tulis.
	
	\item Isi kata pengantar berisi gambaran umum isi dari karya tulis yang disusun. Biasanya dimulai dengan menjelaskan permasalahan atau latar belakang lalu dilanjutkan dengan solusi yang nantinya akan dibahas dalam karya tulis. 
	\item Bagian penutup kata pengantar biasanya berisi permohonan maaf dan harapan penulis tentang karya tulis yang telah dibuat. Setiap karya selalu ada kekurangannyanya, maka permohonan maaf disampaikan atas kekurangan dalam menyusun karya tulis.	Penulis juga sebaiknya menuliskan harapan tentang manfaat karya tulis yang dibuat serta kritik dari pembaca untuk perbaikan penulisan berikutnya
\end{enumerate}
\newpage
Contoh penulisan ucapan terima kasih adalah sebagai berikut :\\
\begin{quote}\color{red}
	Dalam melakukan penelitian ini, penulis mendapat banyak bantuan dan dorongan dari berbagai pihak, diantaranya:
	\begin{itemize}
		\item \pembimbingsatu{} dan \pembimbingdua{} selaku dosen pembimbing Tugas Akhir di \prodis{} \tekm{} \unpar.
		\item Bapak dan ibu {\textit{isikan dengan nama orang tua Anda}}, sebagai orangtua penulis. Terima kasih atas semua kasih sayang, perhatian dan dorongan kepada penulis. 
		\item {Nama rekan} yang telah memberikan dorongan, perhatian serta diskusi yang sangat bermanfaat saat mengerjakan Tugas Akhir ini. 
		\item Rekan-rekan di Lab A1, ............
	\end{itemize}
\end{quote}


Kata Pengantar ditulis dengan menggunakan Bahasa Indonesia formal dan sesuai dengan kaidah penulisan Bahasa Indonesia yang baik dan benar. Pada bagian ucapan terima kasih, hindari penggunaan kata-kata yang berkonotasi negatif meskipun itu adalah nama panggilan akrab kepada teman Anda. Gunakan pula  ejaan nama yang benar. Kata pengantar maksimal 2 halaman. \\

Terima kasih atas pengertian dan kerja sama Anda. Besar harapan kami, \lta{} Anda akan sangat berguna bagi perkembangan \prodis{} \tekm{} \unpar{} pada khususnya serta khazanah keilmuan Teknik Mekatronika pada umumnya. 




   % File Kata Pengantar
% Table of contents
\clearpage
\phantomsection
\addcontentsline{toc}{chapter}{Daftar Isi}
\begin{center}
	\textbf{\large  Daftar Isi}\\[3em]
\end{center}
\renewcommand{\cftdotsep}{\cftnodots}
\setlength{\cftbeforetoctitleskip}{-0.5cm}
\renewcommand{\cfttoctitlefont}{\hfill\large\bfseries}
\renewcommand{\cftaftertoctitle}{\hfill}
\renewcommand\contentsname{}
\tableofcontents


     % File Daftar Isi
% List of table
\clearpage
\phantomsection
\addcontentsline{toc}{chapter}{Daftar Tabel}
%\renewcommand{\cftloftitlefont}{\hfill\large\bfseries}
%\setlength{\cftbeforeloftitleskip}{-0.5cm}
%\renewcommand{\cftafterloftitleskip}{\hfill}
%\renewcommand\listtablename{\centerline {\large\bfseries  DAFTAR TABEL}}
\begin{center}
{\bfseries \large Daftar Tabel}
\end{center}
\vspace{2em}

\daftartabel

     % File Daftar Tabel
% List of fig
\clearpage
\phantomsection
\addcontentsline{toc}{chapter}{Daftar Gambar}
\setlength{\cftbeforeloftitleskip}{-0.5cm}
\renewcommand{\cftloftitlefont}{\hfill\large\bfseries}
\renewcommand{\cftafterloftitle}{\hfill}
\renewcommand\listfigurename{Daftar Gambar}

\listoffigures

     % File Daftar Gambar
\clearpage
\phantomsection
\addcontentsline{toc}{chapter}{Daftar Simbol}
\begin{center}
{\bfseries \large Daftar Simbol dan Variabel}
\end{center}
\vspace{3em}


\begin{tabbing}
AAAA \=BBBBBB \=XX \=YYYYYY \=Z \kill
\>$\approx$ \>\> mendekati \\
\>$\equiv$ \>\> sama dan identik \\
\>$\forall$ \>\> untuk semua \\
\>$\in$ \>\> elemen matrik \\
\>$\sim$ \>\> sebanding \\
\>$\triangleq$ \>\> didefinisikan sebagai \\
\>$i$ \>\> bilangan imajiner \\
\>$\hbar$ \>\> konstanta Planck \\
\>$H$ \>\> matrik Hamiltonian \\
\>$r_i$ \>\> koordinat vektor posisi dari  partikel\\
\>$N$ \>\> jumlah partikel \\
\>$V$ \>\> operator energi potensial \\
\>$T$ \>\> operator energi kinetik \\
\>$n_g$ \>\> kerapatan elektron pada keadaan dasar \\
\>$\Psi$ \>\> persamaan gelombang \\
\>$E$ \>\> energi \\
\>$v_{eff}$ \>\>  potensial efektif\\
\>$v_{exc}$ \>\>  potensial  \emph{exchange correlation}\\
\>$E_{xc}$ \>\> energi \emph{exchange correlation} \\
\>$\phi$ \>\> persamaan gelombang pada persamaan Kohn-Sham

\end{tabbing}



    % File Daftar Simbol & Variabel

\mainmatter               % BAGIAN UTAMA DOKUMEN
\pagenumbering{arabic}
\onehalfspacing           % Spasi 1.5
  \chapter{\uppercase{Pendahuluan}}


\section{Latar Belakang Masalah}
\label{sec:latarbelakang} 
Sistem pendulum terbalik adalah sistem yang sering digunakan untuk mendemonstrasikan dinamika sistem dan implementasi sistem kontrol \cite{lundberg2010history}. Sistem pendulum terbalik dapat digunakan sebagai salah satu \textit{benchmark} pada penerapan teori kontrol. \textit{Benchmark} dari teori kontrol yang dimaksud adalah penggunaan model yang telah diturunkan dan melakukan pengukuran validasi terhadap efisiensi serta performansi dari metode kontrol yang diimplementasikan pada sistem yang digunakan \cite{boubaker2012inverted}. Stabilisasi dari batang pada posisi tegak lurus merupakan salah satu \textit{benchmark} dari teori kontrol \cite{jadlovska2012swingup}. 

\begin{figure}[H] %h artinya here!
	\centering
	\includegraphics[width=0.8\textwidth]{SPTL}
	\caption{Ilustrasi Sistem Pendulum Terbalik Linear}
	\label{fig:sptl}
\end{figure}

Sistem pendulum terbalik sendiri adalah sistem non-linear yang dapat dibedakan menjadi dua macam, yaitu linear dan rotari \cite{jadlovska2012swingup}. Sistem pendulum terbalik linear terdiri dari suatu batang yang diletakkan pada sebuah kereta (\textit{cart}) seperti pada gambar \ref{fig:sptl}. Kereta (\textit{cart}) akan digerakkan menggunakan motor yang akan dikendalikan sehingga mengasilkan gerakan ayunan dari batang yang berfungsi untuk menjaga batang dapat mencapai posisi vertikal terhadap bidang horizontal \cite{bakaravc2017design}. 

Dalam beberapa tahun terakhir, terdapat beberapa pengaplikasian dari sistem pendulum terbalik, seperti \textit{Two-Wheeled Self-Balancing Robot} dan kontrol \textit{Rocket Thruster} saat lepas landas \cite{anderson1989learning,ooi2003balancing,hellman2015two}. Metode kontrol pada sistem pendulum terbalik yang dapat digunakan adalah kontrol LQR (\textit{Linear Quadratic Regulator}), Logika Fuzzy, Jaringan Neural dan Kontrol PID.\cite{kumar2012controller,anderson1989learning,hellman2015two,nasir2008performance}


\section{Identifikasi dan Perumusan Masalah.}
\label{sec:identifikasi}
Berdasarkan uraian pada latar belakang masalah di atas, dapat diiidentifikasi masalah untuk menyelesaikan penelitian sistem pendulum terbalik adalah:

\begin{enumerate}[noitemsep]
	\item Apa saja komponen dari sistem pendulum terbalik?
	\item Apa saja komponen dari sistem pendulum terbalik yang perlu dimodelkan?
	\item Bagaimana cara memodelkan sistem pendulum terbalik?
	\item Apa Metode kontrol yang tepat untuk mengendalikan motor agar dapat digunakan simpangan maksimal dari sistem pendulum terbalik yaitu $\theta$ = 180\textsuperscript{o}?
\end{enumerate}

\section{Batasan Masalah dan Asumsi}
\label{sec:batasan}
Agar penelitian Tugas Akhir ini dapat diselesaikan dengan baik, perlu ada batasan pada masalah utama di atas. Batasan masalah tersebut antara lain: 

\begin{enumerate}[noitemsep]
	\item Sistem yang akan dibangun masih berupa rancangan purwarupa (prototype).
	\item Sistem pendulum terbalik akan menggunakan satu buah batang.
	\item Sistem akan dibangun dengan basis Arduino atau Raspberry Pi.
\end{enumerate}

\section{Tujuan \TA}
\label{sec:tujuan-ta}
Tujuan penulisan Tugas Akhir ini adalah membuat model dan mengimplementasikan metode kontrol umpan balik pada sistem pendulum terbalik dan membuat prototipe dari sistem, serta menampilkan perfroma dari sistem pendulum terbalik menggunakan suatu metode kontrol untuk dapat mempertahankan posisi dari batang yang diinginkan (tegak Lurus terhadap bidang horizontal) dengan sudut simpangan $\theta$=180\textsuperscript{o}. 

\section{Manfaat \TA}
\label{sec:manfaat-ta}
Berikut adalah manfaat dari penelitian sistem pendulum terbalik untuk beberapa pihak, antara lain: 
\begin{itemize}[noitemsep]
	\item Laboratorium Kontrol yang ingin memperlihatkan implementasi teori kontrol pada suatu sistem (pada kasus ini Sistem Pendulum Terbalik).
	\item Pembaca yang ingin mempelajari pemanfaatan teori kontrol pada sistem pendulum terbalik.
	\item Penelitian pribadi, untuk menambah pengetahuan dan pengalaman menyelesaikan masalah nyata di lapangan.
	\item Pengembangan ilmu pengetahuan, terutama pada bidang sistem kontrol, dan sistem pengukuran dan akuisisi data.
\end{itemize}

\section{Metodologi \TA}
\label{sec:metodologi-ta}
Metodologi yang dilakukan pada penelitian ini yaitu metode eksperimen yang diawali dengan studi literatur terlebih dahulu. Mempelajari topik penelitian yang akan dikerjakan, mencari solusi dari masalah, mempelajari seluruh komponen yang akan digunakan, dan mempelajari cara memodelkan dan mengendalikan sistem pendulum terbalik.

\section{Sistematika Penulisan}
Laporan \lta{} ini dibagi menjadi 3 bab, yakni sebagai berikut:
\begin{enumerate}
\item \textbf{Bab 1 Pendahuluan}. Dalam bab ini dijelaskan mengenai latar belakang masalah, identifikasi dan perumusan masalah, batasan masalah dan asumsi, tujuan Tugas Akhir, manfaat Tugas Akhir, metodologi Tugas Akhir serta sistematika penulisan \lta.
\item \textbf{Bab 2 Tinjauan Pustaka}. Bab ini berisi teori-teori yang berhubungan dengan pemecahan masalah dan dibutuhkan dalam pengolahan data serta analisis. Teori-teori dasar ini diperoleh melalui proses telaah pustaka yang intensif pada sejumlah pustaka yang direkomendasikan oleh dosen pembimbing, seperti misalnya: teori rangkaian listrik, teori sistem digital yang sesuai, teori tentang pengendali (mikroprosesor, arduino, raspberry Pi atau PLC), teori pengukuran dan akuisisi data, cara kerja sensor yang digunakan dan aktuator yang dibutuhkan dalam rancangan sistem mekatronika.
\item \textbf{Bab 3 Perancangan Sistem}. Dalam bab ini dipaparkan antara lain:
	\begin{enumerate}
		\item Kriteria/ spesifikasi produk/sistem yang Anda usulkan.
		\item Usulan disain untuk menyelesaikan masalah yang telah dipaparkan di bab sebelumnya (Bab 1). Pada bagian ini, usulan disain dituliskan hingga detil. 
		\item Proses/prosedur pembuatan disain produk/sistem.
		\item Rencana pengujian produk/sistem.
	\end{enumerate}
\end{enumerate}	      % File BAB 1: PENDAHULUAN
  \chapter{\uppercase{Tinjauan Pustaka}}
\onehalfspacing
Bab ini berisi teori-teori yang berhubungan dengan pemecahan masalah dan dibutuhkan dalam pengolahan data serta analisis pada penelitian \TA{} ini. Teori-teori dasar ini diperoleh melalui proses telaah pustaka yang intensif pada sejumlah pustaka yang direkomendasikan oleh dosen pembimbing, seperti misalnya: teori rangkaian listrik, teori sistem digital yang sesuai, teori tentang pengendali (mikroprosesor, Arduino, Raspberry Pi atau PLC), teori pengukuran dan akuisisi data, cara kerja sensor yang digunakan dan aktuator yang dibutuhkan dalam rancangan sistem mekatronika.

Tinjauan literatur harus dapat meringkas apa yang sudah diketahui hingga saat ini (\textit{state of the art}), merinci konsep-konsep kunci dan faktor-faktor utama atau
parameter dan hubungan yang mendasarinya, menggambarkan setiap pendekatan yang ada yang saling melengkapi, menyebutkan ketidakkonsistenan atau kekurangan dalam karya yang diterbitkan, mengidentifikasi hasil yang dilaporkan yang tidak meyakinkan atau bertentangan, dan memberikan alasan yang kuat untuk melakukan penelitian lebih lanjut.

Hasil penelitian sebelumnya yang terkait dengan penelitian yang sedang dilakukan juga diletakkan pada bab Tinjauan Pustaka. Bab ini diletakkan sesudah Bab Pendahuluan dan disebut sebagai Bab 2. Bagian ini merupakan perantara antara Judul Bab dan Judul sub-bab pertama; bagian ini berisikan ringkasan dari Bab 2. Sebagai bagian pengantar untuk Bab 2, ceritakan apa yang dituliskan dalam Bab 2 ini, maksimal $\nicefrac{1}{4}$ halaman.

Berikut ini contoh teori yang biasa digunakan pada penelitian di Teknik Mekatronika.

\section{Hukum-hukum Rangkaian Listrik}
Beberapa hukum utama yang harus diketahui oleh seorang calon sarjana Teknik Mekatronika antara lain adalah:
\begin{enumerate}
	\item Hukum Ohm
	\item Hukum Kirchoff
	\item dst \ldots
\end{enumerate} 
\subsection{Hukum Ohm}
Salah satu hukum utama yang harus diketahui oleh seorang calon sarjana Teknik Mekatronika adalah hukum Ohm. Pada tahun 1827, Georg Simon Ohm (1789-1854), seorang ahli fisika Jerman, menemukan  hubungan langsung yang bersifat proporsional antara tegangan yang muncul di kedua terminal sebuah resistor dengan arus yang melaluinya. Hubungan inilah yang sekarang dikenal sebagai Hukum Ohm, yang secara matematis dapat dituliskan sebagai:
\begin{equation}
v=iR
\label{eq:hkOhm}
\end{equation}
dst \ldots

\subsection{Hukum Kirchoff}
Ketika rangkaian elektronika bertambah rumit, maka Hukum Ohm tidak mencukupi lagi untuk melakukan analisis. Perlu hukum lain untuk melengkapi analisis. Pada tahun 1847, seorang fisikawan Jerman bernama Gustav Robert Kirchhoff (1824-1887) menerbitkan 2 buah hukum untuk membantu analisis, yang dikenal sebagai Hukum Arus Kirchhoff (\textit{Kirchhoff's Current Law}, KCL) dan Hukum Tegangan Kirchhoff (\textit{Kirchhoff's Voltage Law}, KVL).\\

Hukum pertama Kirchhoff (KCL) menyatakan:
\begin{quote}
	Penjumlahan arus yang menuju sebuah \textit{node} (atau sebuah lingkungan tertutup) adalah sama dengan nol.\cite{John03a}
\end{quote}
Secara matematis dapat dituliskan sebagai:
\begin{equation}
\sum_{n=1}^N i_n =0
\label{eq:KCL}
\end{equation}
Ilustrasi untuk hukum KCL ini dapat dilihat pada Gambar \ref{ill:kcl}.
\begin{figure}[h!]
	\begin{center}
		\begin{circuitikz} [scale=1]
			%short circuit
			\draw (0,-0.25) to [short, i_=$i_1$] (2,0)
			to [short, i_=$i_2$] (1,1);
			\draw (3.5,-0.25) to [short, i^=$i_4$,-*] (2,0)
			to [short, i_=$i_5$] (2,-1);
			\draw (2,0) to [short, i_=$i_3$] (3,1);
		\end{circuitikz}		
	\end{center}
	\caption{Ilustrasi KCL pada \textit{node}.}
	\label{ill:kcl}
\end{figure}
Persamaan yang sesuai bagi ilustrasi pada Gambar \ref{ill:kcl} adalah:
\begin{equation}
\begin{split}
i_1-i_2-i_3+i_4-i_5=0 \\
i_1+i_3=i_2+i_4+i_5.
\end{split}	
\label{kcl1}
\end{equation}



\section{Menyisipkan Persamaan} 

Beberapa contoh cara menyisipkan persamaan.


\subsection{Membuat Persamaan (\textit{Equation})}
Untuk membuat persamaan di baris yang sama, dapat menggunakan tanda 'dollar' (\verb|$|) dan dilanjutkan dengan persamaannya sendiri. Contoh, persamaan berikut ini: \verb|$\theta(\vec{r}_1,...,\vec{r}_N)$| yang akan menghasilkan $\theta(\vec{r}_1,...,\vec{r}_N)$. Untuk menuliskan karakter khusus seperti huruf \"{o} dalam kata "Schr\"{o}dinger", dapat menggunakan cara penulisan simbol yang umum berlaku di lingkungan \LaTeX.  Anda dapat mencari di internet dengan kata kunci "latex math symbol". Salah satu dokumen tentang simbol matematika di \LaTeX{} dapat diperoleh pada link berikut: \href {https://reu.dimacs.rutgers.edu/Symbols.pdf}{https://reu.dimacs.rutgers.edu/Symbols.pdf}.

Untuk menuliskan persamaan yang berdiri sendiri dan memiliki nomer urut persamaan, gunakan lingkungan \textit{equation}, seperti berikut:
\begin{lstlisting}
\begin{equation}
\begin{split}
i_1-i_2-i_3+i_4-i_5=0 \\
i_1+i_3=i_2+i_4+i_5.
\end{split}	
\end{equation}
\end{lstlisting}

Hasilnya akan menjadi seperti berikut:
\begin{equation}
\begin{split}
i_1-i_2-i_3+i_4-i_5=0 \\
i_1+i_3=i_2+i_4+i_5.
\end{split}	
\end{equation}

\subsection{Menuliskan Matrix}
Contoh berikut ini adalah cara penulisan matrix, dengan ukuran huruf dikecilkan hingga ukuran \verb|\footnotesize|:
{\footnotesize
\begin{equation}
\Psi({\bf r}_1, {\bf r}_2, \cdots {\bf r}_N) = \frac{1}{\sqrt{N!}}\left| \begin{array}{llcl}
\phi_1({\bf r}_1) & \phi_2({\bf r}_1) & \cdots & \phi_N({\bf r}_1)\\
\phi_1({\bf r}_2) & \phi_2({\bf r}_2) & \cdots & \phi_N({\bf r}_2)\\
\phi_1({\bf r}_3) & \phi_2({\bf r}_3) & \cdots & \phi_N({\bf r}_3)\\
\multicolumn{1}{c}{.} & \multicolumn{1}{c}{.} & \multicolumn{1}{c}{.} & \multicolumn{1}{c}{.} \\
\multicolumn{1}{c}{.} & \multicolumn{1}{c}{.} & \multicolumn{1}{c}{.} & \multicolumn{1}{c}{.} \\
\multicolumn{1}{c}{.} & \multicolumn{1}{c}{.} & \multicolumn{1}{c}{.} & \multicolumn{1}{c}{.} \\
\phi_1({\bf r}_N) & \phi_2({\bf r}_N) & \cdots & \phi_N({\bf r}_N)\\
\end{array} \right|
\end{equation}
}


\section{Menuliskan Referensi dan Sitasi}
Jika Anda menuliskan sebuah kutipan dari sebuah referensi, Anda harus menuliskan dari mana kutipan tersebut diambil/dikutip. Gunakan perintah \verb|\cite| \verb|{bib_id}|, setelah kutipan tersebut. \lta{} pada \prodis{} \tekm{} UNPAR mengguna-kan standar sitasi dan penulisan referensi dari IEEE (\ieee)\cite{Helm53}. Dengan menggunakan standar \textit{style} IEEE (ieeetr), Anda dapat lebih praktis menuliskan sumber sitasi, misalkan \cite{John03a, John03b}. Jika lebih banyak dari 2 sumber, \textit{style} ini dapat dituliskan \cite{Thev83a,Thev83b,Norton26,mayer26,purcell13,alex07} dan sebagainya. 

Penulisan referensi pada Daftar Pustaka tidak diurutkan berdasar alfabet, namun berdasar urutan kemunculannya di bagian teks utama dokumen ini. Gunakan mesin pencari Google Scholar untuk mencari referensi dan gunakan fasilitas '\textit{cite}' (\textbf{Gambar \ref{fig:cite}}) dan BibTeX (\textbf{Gambar \ref{fig:bibtex}}) yang ada.

\begin{figure}[h!]
	\centering
	\includegraphics[width=0.8 \textwidth]{cite}
	\caption{Tanda 'cite' pada mesin pencari Google Scholar.}
	\label{fig:cite}
\end{figure}

\begin{figure}[h!]
	\centering
	\includegraphics[width=0.75 \textwidth]{cite2}
	\caption{\textit{Icon} BibTeX pada hasil 'cite' pada mesin pencari Google Scholar.}
		\label{fig:bibtex}
\end{figure}
\textbf{Catatan:} sitasi pada subbab ini hanya contoh, tidak menunjukkan topik yang sebenarnya.



\section{Menampilkan Gambar}

Untuk gambar yang digunakan pada \lta{} ini bisa menggunakan tipe apa aja, namun disarankan menggunakan file gambar dengan ekstensi \verb|.eps|. File gambar dengan ekstensi \verb|.eps| memiliki resolusi yang baik dan halus. Anda boleh juga mengunakan file gambar dengan ekstensi \verb|.png|, yang memiliki fitur transparansi. Jika Anda menggunakan \LaTeX, disarankan untuk tidak menggunakan file gambar dengan ekstensi \verb|.tif| atau \verb|gif|.

\subsection{Gambar Tipe Satu}
Satu gambar dan terletak di tengah. dapat dilihat pada \textbf{Gambar \ref{fig:nao}}
\begin{figure}[H] %h artinya here!
\centering
\includegraphics[width=0.8\textwidth]{shakenao}
\caption{Tampak muka robot Aldebaran Nao.}
\label{fig:nao}
\end{figure}

Cara melakukan sitasi gambar adalah seperti ini: pada \textbf{Gambar \ref{fig:nao}}, dengan cara men-\textit{cite} di \textit{caption}-nya.

\subsection{Gambar Tipe Dua}
Dua gambar dengan dua \textit{caption} terpisah, dapat dilihat pada \textbf{Gambar \ref{fig:nao1}} dan \textbf{Gambar \ref{fig:nao2}}
\begin{figure}[h!]
\centering
\begin{minipage}{0.35\linewidth}
\centering
\includegraphics[width=0.5\linewidth]{nao}
\caption{Nao kiri}
\label{fig:nao1}
\end{minipage}
\hspace{0.2\linewidth}
\begin{minipage}{0.35\linewidth}
\centering
\includegraphics[width=0.5\linewidth]{nao}
\caption{Nao kanan.}
\label{fig:nao2}
\end{minipage}
\end{figure}

\subsection{Gambar Tipe Tiga}
Contoh dua gambar dengan satu \textit{caption}, dapat dilihat pada \textbf{Gambar \ref{naonao}}.
\begin{figure}[H]
\centering
\subfigure[]{
 \includegraphics[width=0.36\linewidth]{naored}
 \label{nao-1}
 }\hspace{0.1\linewidth}
\subfigure[]{
 \includegraphics[width=0.3\linewidth]{nao}
 \label{nao-2}
}
\caption{Variasi warna dari robot Nao. \subref{nao-1} adalah Nao merah dan \subref{nao-2} adalah Nao biru. Kedua robot ini dibuat oleh Aldebaran Robotics yang berpusat di Perancis.}
\label{naonao}
\end{figure}   

\subsection{Gambar Tipe Empat}
Satu \textit{caption} dengan banyak gambar, dapat dilihat pada \textbf{Gambar \ref{layers}}. {\color{red}\itshape\lipsum[2]}
\begin{figure}[h]
\centering
\subfigure[]{
\includegraphics[width=0.15\linewidth]{naored}
\label{fig:1}
}\hspace{0.1\linewidth}
\subfigure[]{
\includegraphics[width=0.15\linewidth]{twonao}
\label{fig:2}
}\hspace{0.1\linewidth}
\subfigure[]{
\includegraphics[width=0.15\linewidth]{nao}
\label{fig:3}
}\hspace{0.1\linewidth}
\subfigure[]{
\includegraphics[width=0.15\linewidth]{tienao}
\label{fig:4}
}\hspace{0.1\linewidth}
\subfigure[]{
\includegraphics[width=0.15\linewidth]{nao}
\label{fig:5}
}\hspace{0.1\linewidth}
\subfigure[]{
\includegraphics[width=0.15\linewidth]{naored.jpg}
\label{fig:6}
}\hspace{0.1\linewidth}
\subfigure[]{
\includegraphics[width=0.15\linewidth]{twonao}
\label{fig:7}
}\hspace{0.1\linewidth}
\subfigure[]{
\includegraphics[width=0.15\linewidth]{tienao.jpg}
\label{fig:8}
}
\caption{Beberapa gambar robot Nao.
\subref{fig:1} dan \subref{fig:2} adalah Nao merah dan Nao berdua; 
\subref{fig:3} dan \subref{fig:4} adalah Nao biru dan Nao merah di dalam lingkaran;
\subref{fig:5} dan \subref{fig:6} adalah Nao biru dan Nao merah;
\subref{fig:7} dan \subref{fig:8} adalah Nao berdua dan Nao merah di dalam lingkaran.
}
\label{layers}\end{figure} 


\section{Menuliskan Tabel}

Tabel pada \lta{} ini dituliskan dengan format rata-tengah (\textit{centered alignment}) dan penulisan \textit{caption} di atas tabel. Contoh penulisan tabel dapat dilihat pada \textbf{Tabel \ref{tab:random}} berikut ini.

\begin{table}[H]
\caption{Contoh pertama}
\label{tab:random}
\begin{center}
\setlength\extrarowheight{5pt}	
\begin{tabular}{|c|c|c|c|}	
\hline
\textbf{$Title_1$} & \textbf{$title_{\mathrm{DUA}}$} & \textbf{Title 3 } & \textbf{Title 4}  \\
\hline
1647 &  1.97  &  0.68 &  1.90 \\
2301 &  2.92  &  1.06 &  2.75 \\
2969 &  3.23  &  1.16 &  3.78 \\
4625 &  6.72  &  1.87 &  5.59 \\
\hline
\end{tabular}
\end{center}
\end{table}

\textbf{Tabel \ref{tab:random}} di atas adalah tabel random. {\color{red}\itshape\lipsum[2]}

%%%%%%%%%%%%%%%%%%%%%%%%%%%%%%%%%%%%%%%%%%%%%%%%%%
% Keep the following \cleardoublepage at the end of this file, 
% otherwise \includeonly includes empty pages.
\cleardoublepage	      % File BAB 2: TINJAUAN PUSTAKA
  \chapter{\uppercase{Perancangan Sistem}}
{\color{red}\itshape\lipsum[1]}

\section{Spesifikasi Sistem/Disain}
{\color{red}\itshape\lipsum[2]}
\begin{table}[H]
	\caption{Contoh kedua}
	\label{tab:random1}
	\begin{center}  
		\setlength\extrarowheight{5pt}	
		\begin{tabular}{|c|c|c|c|}	
			\hline
			\textbf{$Title_1$} & \textbf{$title_{\mathrm{DUA}}$} & \textbf{Title 3 } & \textbf{Title 4}  \\
			\hline
			1647 &  1.97  &  0.68 &  1.90 \\
			2301 &  2.92  &  1.06 &  2.75 \\
			2969 &  3.23  &  1.16 &  3.78 \\
			4625 &  6.72  &  1.87 &  5.59 \\
			\hline
		\end{tabular}
	\end{center}
\end{table}

{\color{red}\itshape\lipsum[3]}
\section{Rincian Disain}
\subsection{Komponen Utama}
Pada bagian ini, Anda tampilkan disain/sistem utama Anda. Tampilkan rancangan bagian/komponen yang penting saja. Bagian/ skema lebih rinci dapat Anda letakkan pada bagian Lampiran. 

\subsection{Komponen lebih rinci}
Bagan atau skema disain yang berukuran lebih dari ukuran kertas A4 dapat dicetak menggunakan kertas berukuran lebih besar dari A4. Pada saat dikumpulkan, skema ini harus dilipat menjadi berukuran A4. Perhatikan juga bahwa ada ukuran \textit{margin} kiri untuk dokumen ini, sebesar 4 cm. Letakkan gambar skema ini pada bagian Lampiran.\\ 
{\color{red}\itshape\lipsum[4]}

\section{Rencana Pembuatan}
Dalam bagian ini dijabarkan prosedur/rencana kerja/urutan pembuatan dari sistem/disain yang diusulkan. Tampilkan pula jadwal rencana pembuatan disain Anda, menggunakan diagram Gantt Chart. Tambahkan beberapa titik \textit{capstone} (titik capaian) pada rencana Anda. \textit{Capstone} ini penting saat Anda mulai mengerjakan disain, saat Tugas Akhir II nantinya. Hal ini akan membiasakan Anda untuk memenuhi target capaian tertentu pada suatu proses.\\

Bila Gantt Chart Anda terlalu panjang, maka Anda dapat meletakkan Gantt Chart ini pada bagian Lampiran.\\

\begin{figure}
	\centering 
	\includegraphics[width=\textwidth]{gambar/gantt1}
	\caption{Contoh Gantt Chart.}
\end{figure}   
%{\color{red}\itshape\lipsum[5]}

\section{Rencana Pengujian Sistem}
Pada bagian ini, dijabarkan tentang bagaimana Anda akan menguji performa dari disain Anda. Karena hasil disain/simulasi/produk Anda merupakan jawaban dari sebuah masalah, maka perhatikan variabel/ parameter disain awal yang ada pada masalah awal. Contoh dari hal ini antara lain: ukuran/dimensi alat yang dihasilkan, kecepatan perhitungan (pada kasus simulasi), ketepatan mencapai tujuan (akurasi) dan lain sebagainya.
\section{Rincian Biaya}
Jika hasil akhir Tugas Akhir Anda berupa sebuah produk yang memerlu-kan biaya pembuatan atau pembelian material, maka Anda dapat meletakkan tabel kebutuhan biaya pada bagian ini. \prodis{} \tekm{} UNPAR tidak menanggung biaya pembelian material dan proses pemesinan dari disain Anda. Oleh karena itu, sesuaikan disain Anda dengan kemampuan finansial Anda. Jika disain Anda memang '\textit{marketable}', maka disarankan untuk mencari '\textit{investor} untuk proyek Anda. Walau begitu, Program Studi tetap dapat menyediakan/ mengadakan \textbf{sebagian} komponen yang diperlukan, sepanjang komponen tersebut memang telah dianggarkan dalam Rencana Anggaran Program Studi.\\

Hak cipta dan hasil produk Tugas Akhir sepenuhnya menjadi milik \prodis{} \tekm{} \unpar.  \\

{\color{red}\itshape\lipsum[4]}


	      % File BAB 3: PERANCANGAN SISTEM
  
 \backmatter              % BAGIAN AKHIR DOKUMEN
 \clearpage
\phantomsection
\addcontentsline{toc}{chapter}{Daftar Pustaka}
\renewcommand{\refname}{}
\begin{center}
\textbf{\Large Daftar Pustaka}\\[4.5em]
\end{center}
 \bibliographystyle{ieeetr} 
  \bibliography{dta1-pustaka}

     % File Daftar Pustaka (Bibliography)
 
%%%%%%%%%%%%%%%%%%%%%%%%%%% BAGIAN LAMPIRAN 
\begin{appendices}
\newpage
\thispagestyle{empty}
%\vspace{5cm}
\vspace*{\fill}
\begin{center}
	\huge \textbf{Lampiran A}\\
	\vspace{10mm}
	{\Large 
	1. \textit{Graph Theory Notations}\\
	2. \textit{Social Force Model}\\
	}

\end{center}
\vspace*{\fill}
%\appendix
\linespread{1.25} % Spasi 1.5
\chapter{Beberapa Teori Pendukung}
\section{\textit{Graph Theory Notations}}

It is natural to model information exchange among vehicles by directed or undirected graphs.\textsuperscript{1} Suppose that a team consists of p vehicles. A \textit{directed
graph} is a pair $(\mathcal{V}_p,  \mathcal{E}_p)$, where $\mathcal{V}_p = {1, \ldots , p}$ is a finite nonempty \textit{node} set and $\mathcal{E}_p \subseteq \mathcal{V}_p \times \mathcal{V}_p$ is an \textit{edge} set of ordered pairs of nodes, called \textit{edges}. The edge $(i, j)$ in the edge set of a directed graph denotes that vehicle $j$ can obtain information from vehicle $i$, but not necessarily \textit{vice versa}. Self-edges $(i, i)$ are not allowed unless otherwise indicated. For the edge $(i, j)$, $i$ is the \textit{parent node} and $j$ is the \textit{child node}. In contrast to a directed graph, the pairs of nodes in an \textit{undirected graph} are unordered, where the edge $(i, j)$ denotes that vehicles $i$ and $j$ can obtain information from each other. Note that an undirected graph can be viewed as a special case of a directed graph, where an edge $(i, j)$ in the undirected graph corresponds to edges $(i, j)$ and $(j, i)$ in the directed graph. If an edge $(i, j) \in \mathcal{E}_p$, then node $i$ is a \textit{neighbor} of node $j$. The set of neighbors of node $i$ is denoted as $\mathcal{N}_i$. A \textit{weighted graph} associates a weight with every edge in the graph. In this book, all graphs are weighted. The \textit{union} of a collection of graphs is a graph whose node and edge sets are the unions of the node and edge sets of the graphs in the collection.


\begin{tikzpicture}[remember picture,overlay]
\node[xshift=40mm,yshift=-48mm,anchor=north west] at (current page.north west){%
	\includegraphics[width=\textwidth]{gambar/sample}};
\end{tikzpicture}


%\newpage
\chapter{Listing dan Skema Sistem}
\section{Listing Program Sistem XYZ}
\vspace{-10mm}
\begin{lstlisting}
void setup() {
// Open serial communications and wait for port to open:
Serial.begin(9600);
while (!Serial) {
; // wait for serial port to connect. 
Needed for native USB port only
}

Serial.print("Initializing SD card...");

if (!SD.begin(4)) {
Serial.println("initialization failed!");
while (1);
}
Serial.println("initialization done.");

root = SD.open("/");

printDirectory(root, 0);

Serial.println("done!");
}

void loop() {
// nothing happens after setup finishes.
}



\end{lstlisting}
%\newpage
%\thispagestyle{empty}
\section{Layout Rangkaian Listrik Sistem XYZ}
\vspace{5mm}
%% Gunakan env. sideways untuk membuat gambar tercetak landscape.
\vfill
\begin{figure}[ht]
	\centering
	\begin{sideways}
		\begin{minipage}{\textwidth}
		\includegraphics[scale=0.8]{maxflash}
		\end{minipage}
	\end{sideways}
	
	\caption{Rangkaian elektronik XYZ.}
\end{figure}
\vfill
%\newpage
\section{Desain Mekanik Sistem XYZ}

\vfill
\begin{figure}[ht]
\begin{minipage}{\textwidth}
\centering	
\includegraphics[width=0.85\linewidth]{mechdraw} % Sesuaikan nilai width dengan gambar anda.
\end{minipage}
\caption{Komponen A.}
\end{figure}
\vfill

 % Cover pembatas untuk Lampiran A
% Isi Lampiran A
\appendix
\linespread{1.25} % Spasi 1.5
\chapter{Beberapa Teori Pendukung}
\section{\textit{Graph Theory Notations}}

It is natural to model information exchange among vehicles by directed or undirected graphs.\textsuperscript{1} Suppose that a team consists of p vehicles. A \textit{directed
graph} is a pair $(\mathcal{V}_p,  \mathcal{E}_p)$, where $\mathcal{V}_p = {1, \ldots , p}$ is a finite nonempty \textit{node} set and $\mathcal{E}_p \subseteq \mathcal{V}_p \times \mathcal{V}_p$ is an \textit{edge} set of ordered pairs of nodes, called \textit{edges}. The edge $(i, j)$ in the edge set of a directed graph denotes that vehicle $j$ can obtain information from vehicle $i$, but not necessarily \textit{vice versa}. Self-edges $(i, i)$ are not allowed unless otherwise indicated. For the edge $(i, j)$, $i$ is the \textit{parent node} and $j$ is the \textit{child node}. In contrast to a directed graph, the pairs of nodes in an \textit{undirected graph} are unordered, where the edge $(i, j)$ denotes that vehicles $i$ and $j$ can obtain information from each other. Note that an undirected graph can be viewed as a special case of a directed graph, where an edge $(i, j)$ in the undirected graph corresponds to edges $(i, j)$ and $(j, i)$ in the directed graph. If an edge $(i, j) \in \mathcal{E}_p$, then node $i$ is a \textit{neighbor} of node $j$. The set of neighbors of node $i$ is denoted as $\mathcal{N}_i$. A \textit{weighted graph} associates a weight with every edge in the graph. In this book, all graphs are weighted. The \textit{union} of a collection of graphs is a graph whose node and edge sets are the unions of the node and edge sets of the graphs in the collection.


\begin{tikzpicture}[remember picture,overlay]
\node[xshift=40mm,yshift=-48mm,anchor=north west] at (current page.north west){%
	\includegraphics[width=\textwidth]{gambar/sample}};
\end{tikzpicture}


%
\section{\textit{Social Force Model}}
\singlespacing
Dinamika pergerakan manusia telah menarik perhatian para ilmuwan sejak dekade 70-an. Berbagai pendekatan dilakukan untuk dapat memodelkan perilaku pergerakan manusia ini, mulai dari pendekatan statistik (Henderson, 1971), fluida (Helbing 1992.). hingga model pergerakan partikel gas (Helbing dkk., 1995). Dari berbagai macam model tersebut, dinamika pergerakan kerumunan manusia dapat dibagi menjadi 2 model besar, yaitu model makroskopik dan model mikroskopik. Model makroskopik lebih dahulu muncul dengan menganalogikan kerumunan manusia sebagai satu kesatuan entitas yang memiliki energi yang bergerak bersama. Dalam kelompok model ini dikenal antara lain model Hughes, Daamen- Hoogendoorn-Bovy, Bruno dkk serta model Colombo-Rosini (Kormanova, 2014).\\

% CATATAN AKHIR 17 Feb 2019 !!
% Tambahkan beberapa reference ke dta-pustaka.bib

\begin{figure}[h!]
	\centering
	\includegraphics[width=0.6\textwidth]{sfm1}
	\caption{Gaya yang bekerja pada seseorang saat berjalan.}
\end{figure}

Sedangkan model mikroskopik mulai muncul sejalan dengan perkembangan
teknologi komputer yang makin maju, karena model ini memerlukan perhitungan yang sangat masif. Pada model mikroskopik, dinamika pergerakan manusia dianggap sebagai sebuah hasil interaksi gaya yang ditimbulkan oleh masingmasing individu, sehingga dianggap lebih realistik pemodelannya. Beberapa model mikroskopik antara lain model SFM (Helbing \& Molnar, 1995), model \textit{cellular automata} (Blue, V. J dan Adler, J. L., 1999 \& 2000) dan model berbasis \textit{artificial intelligence} (Bonabeau dkk., 1999).\\

Dari berbagai macam model yang ada, pada penelitian ini dipilih model SFM
dengan pertimbangan bahwa :
\begin{enumerate}
	\item SFM relatif sederhana dibandingkan dengan model perilaku pergerakan
	pejalan kaki yang lain, namun mampu mewakili perilaku yang umum terjadi saat manusia berjalan di dalam sebuah kerumunan.
	\item SFM mampu menggambarkan ruang pribadi masing-masing individu yang
	dimodelkan.
	\item Memodelkan perilaku pergerakan individu yang mengikuti individu lain
	pada jarak yang aman dan nyaman.
	\item Memodelkan perilaku pergerakan individu yang menghindari tembok atau
	halangan lain pada jarak tertentu.
\end{enumerate}

Dinamika pergerakan individu ini dapat dituliskan dalam sebuah persamaan dinamika tertentu. Persamaan gerak dinamika untuk seluruh individu \textit{pedestrian} yang berada di dalam kerumunan dapat dituliskan sebagai berikut:

\begin{equation}
\begin{split}
F_i(t) & = F_i^0(v_i,v_i^0e_i) + \sum_{j}F_{ij}(e_i,r_i-r_j) \\
 & + \sum_{O}F_{iO}(e_i,r_i-r_O^i) + \sum_{a}F_{ia}(e_i,r_i-r_k,t)
\end{split}
\end{equation} 





\newpage
\thispagestyle{empty}
%\vspace{5cm}
\vspace*{\fill}
\begin{center}
	\huge \textbf{Lampiran B}\\
	\vspace{10mm}
	{\Large 
	1.	\textit{Listing} Program\\
	2. \textit{Layout} Rangkaian Listrik Sistem XYZ\\
	3. Desain Mekanik Sistem XYZ\\
	4. Diagram Skema Grumman Sailboat\\
	}

\end{center}
\vspace*{\fill}
%\appendix
\linespread{1.25} % Spasi 1.5
\chapter{Beberapa Teori Pendukung}
\section{\textit{Graph Theory Notations}}

It is natural to model information exchange among vehicles by directed or undirected graphs.\textsuperscript{1} Suppose that a team consists of p vehicles. A \textit{directed
graph} is a pair $(\mathcal{V}_p,  \mathcal{E}_p)$, where $\mathcal{V}_p = {1, \ldots , p}$ is a finite nonempty \textit{node} set and $\mathcal{E}_p \subseteq \mathcal{V}_p \times \mathcal{V}_p$ is an \textit{edge} set of ordered pairs of nodes, called \textit{edges}. The edge $(i, j)$ in the edge set of a directed graph denotes that vehicle $j$ can obtain information from vehicle $i$, but not necessarily \textit{vice versa}. Self-edges $(i, i)$ are not allowed unless otherwise indicated. For the edge $(i, j)$, $i$ is the \textit{parent node} and $j$ is the \textit{child node}. In contrast to a directed graph, the pairs of nodes in an \textit{undirected graph} are unordered, where the edge $(i, j)$ denotes that vehicles $i$ and $j$ can obtain information from each other. Note that an undirected graph can be viewed as a special case of a directed graph, where an edge $(i, j)$ in the undirected graph corresponds to edges $(i, j)$ and $(j, i)$ in the directed graph. If an edge $(i, j) \in \mathcal{E}_p$, then node $i$ is a \textit{neighbor} of node $j$. The set of neighbors of node $i$ is denoted as $\mathcal{N}_i$. A \textit{weighted graph} associates a weight with every edge in the graph. In this book, all graphs are weighted. The \textit{union} of a collection of graphs is a graph whose node and edge sets are the unions of the node and edge sets of the graphs in the collection.


\begin{tikzpicture}[remember picture,overlay]
\node[xshift=40mm,yshift=-48mm,anchor=north west] at (current page.north west){%
	\includegraphics[width=\textwidth]{gambar/sample}};
\end{tikzpicture}


%\newpage
\chapter{Listing dan Skema Sistem}
\section{Listing Program Sistem XYZ}
\vspace{-10mm}
\begin{lstlisting}
void setup() {
// Open serial communications and wait for port to open:
Serial.begin(9600);
while (!Serial) {
; // wait for serial port to connect. 
Needed for native USB port only
}

Serial.print("Initializing SD card...");

if (!SD.begin(4)) {
Serial.println("initialization failed!");
while (1);
}
Serial.println("initialization done.");

root = SD.open("/");

printDirectory(root, 0);

Serial.println("done!");
}

void loop() {
// nothing happens after setup finishes.
}



\end{lstlisting}
%\newpage
%\thispagestyle{empty}
\section{Layout Rangkaian Listrik Sistem XYZ}
\vspace{5mm}
%% Gunakan env. sideways untuk membuat gambar tercetak landscape.
\vfill
\begin{figure}[ht]
	\centering
	\begin{sideways}
		\begin{minipage}{\textwidth}
		\includegraphics[scale=0.8]{maxflash}
		\end{minipage}
	\end{sideways}
	
	\caption{Rangkaian elektronik XYZ.}
\end{figure}
\vfill
%\newpage
\section{Desain Mekanik Sistem XYZ}

\vfill
\begin{figure}[ht]
\begin{minipage}{\textwidth}
\centering	
\includegraphics[width=0.85\linewidth]{mechdraw} % Sesuaikan nilai width dengan gambar anda.
\end{minipage}
\caption{Komponen A.}
\end{figure}
\vfill

 % Cover pembatas untuk Lampiran B
% Isi Lampiran B
\newpage
\chapter{Listing dan Skema Sistem}
\section{Listing Program Sistem XYZ}
\vspace{-10mm}
\begin{lstlisting}
void setup() {
// Open serial communications and wait for port to open:
Serial.begin(9600);
while (!Serial) {
; // wait for serial port to connect. 
Needed for native USB port only
}

Serial.print("Initializing SD card...");

if (!SD.begin(4)) {
Serial.println("initialization failed!");
while (1);
}
Serial.println("initialization done.");

root = SD.open("/");

printDirectory(root, 0);

Serial.println("done!");
}

void loop() {
// nothing happens after setup finishes.
}



\end{lstlisting}
\newpage
%\thispagestyle{empty}
\section{Layout Rangkaian Listrik Sistem XYZ}
\vspace{5mm}
%% Gunakan env. sideways untuk membuat gambar tercetak landscape.
\vfill
\begin{figure}[ht]
	\centering
	\begin{sideways}
		\begin{minipage}{\textwidth}
		\includegraphics[scale=0.8]{maxflash}
		\end{minipage}
	\end{sideways}
	
	\caption{Rangkaian elektronik XYZ.}
\end{figure}
\vfill
\newpage
\section{Desain Mekanik Sistem XYZ}

\vfill
\begin{figure}[ht]
\begin{minipage}{\textwidth}
\centering	
\includegraphics[width=0.85\linewidth]{mechdraw} % Sesuaikan nilai width dengan gambar anda.
\end{minipage}
\caption{Komponen A.}
\end{figure}
\vfill

\newpage
%\thispagestyle{empty}
\section{Diagram Grumman Sailboat}
\vspace{5mm}
%% Gunakan env. sideways untuk membuat gambar tercetak landscape.
\vfill
\begin{figure}[ht]
	\centering
	\begin{sideways}
		\begin{minipage}{\textwidth}
		\includegraphics[width=\textwidth]{grum}
		\end{minipage}
	\end{sideways}
	\caption{Skema Grumman Sailboat.}
\end{figure}
\vfill
\end{appendices}

	
\end{document}
