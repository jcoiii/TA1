% Version 1.0.0. (16 Feb 2019)
% ==================================

% Format kertas
\usepackage{geometry}
\geometry{a4paper, top=3cm,bottom=3cm,left=4cm,right=3cm}
%\usepackage{tgtermes} % Times New Roman
%\usepackage{times} % LATEX Roman

% Package yang diperlukan
\usepackage[colorlinks=true, linkcolor =black, anchorcolor = black, citecolor = black, urlcolor = black]{hyperref}
\usepackage[htt]{hyphenat}
\usepackage{listing}	  % Package untuk listing program	
\usepackage{graphicx}     % Package untuk grafik/ gambar
\usepackage{tabularx}     % Package untuk tabel
\usepackage{float}
\usepackage[hang,nooneline,scriptsize,md]{subfigure}
\usepackage{epsfig}	% Package untuk grafik/ gambar *.eps
\usepackage[font=small,labelfont=bf, labelsep=period]{caption}
\usepackage[onehalfspacing]{setspace}      % Package untuk spasi 1.5
\usepackage{indentfirst}
%\setlength{\parindent}{0.6cm}
\usepackage{parskip}
\usepackage[titletoc]{appendix}
\usepackage{cite}          % Package untuk sitasi menggunakan BibTex
\usepackage[indonesian]{babel} % Package untuk Bahasa Indonesia
\usepackage{array,ragged2e}
\usepackage[utf8]{inputenc}
\usepackage{textcase}
\usepackage{bigstrut}
%\usepackage[T1]{fontenc}  Optional, untuk beberapa simbol khusus
\usepackage{rotating}
%\usepackage{relsize}
\usepackage{textcomp} 	% nice greek alphabet
\usepackage{booktabs}
\usepackage{amssymb,amsthm}
\usepackage{amsmath}
\usepackage{tikz} 
\usetikzlibrary{shapes, arrows,calc,bending,matrix}
\usepackage[american, siunitx]{circuitikz}
\usepackage{lipsum}
%\usepackage{graphicx}
\usepackage{bm}
\usepackage{nicefrac}
\usepackage{pdfpages}
%\usepackage[colorlinks=true, linkcolor =black, anchorcolor = black, citecolor = black, urlcolor = black]{hyperref}
\usepackage{courier}
\usepackage{listings}
\lstset{basicstyle=\small\ttfamily,breaklines=true}
\lstset{framextopmargin=50pt}
\usepackage{enumitem}
\usepackage{titlesec}

%%% Pengaturan global
\graphicspath{{gambar/}}   % letak direktori penyimpanan gambar

\pagestyle{plain}


% Format Judul Bab

\renewcommand{\chaptertitlename}{BAB} 
\titleformat{\chapter}[display]
{\bfseries\Large}
{%
	% \vskip-5em
	\filcenter
	\Large\chaptertitlename \ \thechapter} % --> BAB 1
{0mm}
{\filcenter}
\titleformat*{\section}{\bfseries\large}
\titleformat*{\subsection}{\bfseries}
\titlespacing*{\chapter}{0pt}{0pt}{15mm}
% \titlespacing{command}{left spacing}{before spacing}{after spacing}[right]
% Bagian untuk menghilangkan titik2 di daftar isi, gambar, tabel
\usepackage[subfigure]{tocloft}
\renewcommand{\cftpartleader}{\hfill}
\renewcommand{\cftpartafterpnum}{\cftparfillskip}
\renewcommand{\cftchapleader}{\hfill} 
\renewcommand{\cftsecleader}{\hfill}

%tambah "BAB" x di daftar isi 
%\renewcommand*\cftchappresnum{BAB~}
%\renewcommand\chaptername{BAB}
%\settowidth{\cftchapnumwidth}{\cftchappresnum}
%\renewcommand{\cftchapaftersnumb}{\quad}
%\addtocontents{toc}{
%	%\renewcommand\protect*\protect\cftchappresnum{\MakeUppercase{\chaptername}~}}
%	\protect\renewcommand*\protect\cftchappresnum{\chaptername~}}

%tambah "BAB" x di daftar isi 
\renewcommand*\cftchappresnum{BAB~}
\renewcommand\chaptername{BAB}
\settowidth{\cftchapnumwidth}{\cftchappresnum}
\renewcommand{\cftchapaftersnumb}{\quad}
\addtocontents{toc}{
	%\renewcommand\protect*\protect\cftchappresnum{\MakeUppercase{\chaptername}~}}
	\protect\renewcommand*\protect\cftchappresnum{\chaptername~}}

% Penambahan kata "Gambar" dan "Tabel" di daftar gambar dan daftar tabel
\usepackage{chngcntr}  
%\counterwithout{figure}{chapter}
\renewcommand{\cftfigpresnum}{\bfseries Gambar\ }
\renewcommand{\cfttabpresnum}{\bfseries Tabel\ }
\newlength{\mylenf}
% Jarak jeda antara nomer gambar/ tabel dengan keterangannya.
\settowidth{\mylenf}{\cftfigpresnum}
\setlength{\cftfignumwidth}{\dimexpr\mylenf+3em}
\settowidth{\mylenf}{\cfttabpresnum}
\setlength{\cfttabnumwidth}{\dimexpr\mylenf+3em}
\makeatletter

% listoftable tanpa judul
%\counterwithout{table}{chapter}
\newcommand\daftartabel{%
	\phantomsection
	\@starttoc{lot}}
\makeatother


% Menghilangkan judul bibliografi 
\makeatletter
\renewenvironment{thebibliography}[1]{%
	%     \section*{\refname}%
	%      \@mkboth{\MakeUppercase\refname}{\MakeUppercase\refname}%
	\list{\@biblabel{\@arabic\c@enumiv}}%
	{\settowidth\labelwidth{\@biblabel{#1}}%
		\leftmargin\labelwidth
		\advance\leftmargin\labelsep
		\@openbib@code
		\usecounter{enumiv}%
		\let\p@enumiv\@empty
		\renewcommand\theenumiv{\@arabic\c@enumiv}}%
	\sloppy
	\clubpenalty4000
	\@clubpenalty \clubpenalty
	\widowpenalty4000%
	\sfcode`\.\@m}
{\def\@noitemerr
	{\@latex@warning{Empty `thebibliography' environment}}%
	\endlist}
\makeatother


%lampiran
\makeatletter
\newcommand\appendix@chapter[1]{%
	\refstepcounter{chapter}%
	\orig@chapter*{Lampiran \@Alph\c@chapter \\ #1}\vspace{1em} % Format di lembar lampiran 
%	\addcontentsline{toc}{chapter}{LAMP. \@Alph\c@chapter: #1}%
	\addcontentsline{toc}{chapter}{{} \@Alph\c@chapter. #1} % Format di daftar isi
}

\let\orig@chapter\chapter
\g@addto@macro\appendix{\let\chapter\appendix@chapter}
\makeatother

%%% Tambahan format untuk tabel : cell alignment
\newcolumntype{C}[1]{>{\centering\arraybackslash}m{#1}}
\newcolumntype{R}[1]{>{\raggedleft\arraybackslash}m{#1}}
\newcolumntype{L}[1]{>{\raggedright\arraybackslash}m{#1}}
