%%%%%%%%%%%%%%%%%%%%%%%%%%%%%%%%%%%%%%%%%
%Source: 
%Stylish Article
% LaTeX Template
% Version 2.1 (1/10/15)
%
% This template has been modified from:
% http://www.LaTeXTemplates.com
%
% Original author:
% Mathias Legrand (legrand.mathias@gmail.com) 
% With extensive modifications by:
% Vel (vel@latextemplates.com)
%
% Modified to be used in Mechatronics Engineering by
% Dr. Christian Fredy Naa, S.Si., M.Si., M.Sc. 
% Levin Halim, S.T., M.T.
%
% License:
% CC BY-NC-SA 3.0 (http://creativecommons.org/licenses/by-nc-sa/3.0/)
%
%%%%%%%%%%%%%%%%%%%%%%%%%%%%%%%%%%%%%%%%%

%----------------------------------------------------------------------------------------
%	PACKAGES AND OTHER DOCUMENT CONFIGURATIONS
%----------------------------------------------------------------------------------------

\documentclass[fleqn,10pt]{SelfArx} % Document font size and equations flushed left

\usepackage[indonesian]{babel} % Specify a different language here - english by default
\usepackage{lipsum} % Required to insert dummy text. To be removed otherwise
\usepackage{amsmath}
\usepackage{amsfonts}
\usepackage{amssymb}
%----------------------------------------------------------------------------------------
%	COLUMNS
%----------------------------------------------------------------------------------------

\setlength{\columnsep}{0.55cm} % Distance between the two columns of text
\setlength{\fboxrule}{0.75pt} % Width of the border around the abstract

%----------------------------------------------------------------------------------------
%	COLORS
%----------------------------------------------------------------------------------------

\definecolor{color1}{RGB}{0,0,90} % Color of the article title and sections
\definecolor{color2}{RGB}{0,20,20} % Color of the boxes behind the abstract and headings

%----------------------------------------------------------------------------------------
%	HYPERLINKS
%----------------------------------------------------------------------------------------

\usepackage{hyperref} % Required for hyperlinks
\hypersetup{hidelinks,colorlinks,breaklinks=true,urlcolor=color2,citecolor=color1,linkcolor=color1,bookmarksopen=false,pdftitle={Title},pdfauthor={Author}}

%----------------------------------------------------------------------------------------
%	ARTICLE INFORMATION
%----------------------------------------------------------------------------------------

\JournalInfo{Implementasi Kontrol Umpan Balik pada Sistem Pendulum Terbalik} % Nama Mata Kuliah
\Archive{Genap 2018/2019} % Isi dengan tahun ajaran

\PaperTitle{Implementasi Kontrol Umpan Balik pada Sistem Pendulum Terbalik} % Judul tugas akhir
\Authors{Jonathan Chandra\textsuperscript{1}*, Ali Sadiyoko\textsuperscript{2}, Tua Agustinus Tamba\textsuperscript{3}} % Nama pembuat laporan
\affiliation{\textsuperscript{1}\textit{2015630028, Teknik Elektro Konsentrasi Mekatronika, Universitas Katolik Parahyangan, Bandung, Indonesia}} % Afilasi pembuat laporan, NPM, Program Studi, Universitas
\affiliation{\textsuperscript{2}\textit{20000247, Teknik Elektro Konsentrasi Mekatronika, Universitas Katolik Parahyangan, Bandung, Indonesia}} % Afilasi pembuat laporan, NPM, Program Studi, Universitas
\affiliation{*\textbf{E-mail}: 6315028@student.unpar.ac.id, alfa51@unpar.ac.id, ttamba@unpar.ac.id} % Corresponding author

\Keywords{Pendulum Terbalik, Model Dinamika Lagrange, Non-Linier, Kontrol PID} % Kata kunci memuat istilah dan/atau konsep kunci yang dianggap penting. 
\newcommand{\keywordname}{Kata kunci} % Defines the keywords heading name

\addto{\captionsenglish}{\renewcommand{\abstractname}{Abstrak}} %renaming abstract
\addto{\captionsenglish}{\renewcommand{\figurename}{Gambar}} %renaming Figure
\addto{\captionsenglish}{\renewcommand{\tablename}{Tabel}} %renaming Table
\addto{\captionsenglish}{\renewcommand{\refname}{Daftar Pustaka}} %renaming References
\addto{\captionsenglish}{\renewcommand{\contentsname}{Daftar Isi}} %renaming Contents
%----------------------------------------------------------------------------------------
%	ABSTRACT
%----------------------------------------------------------------------------------------

\Abstract{Sistem Pendulum Terbalik adalah salah satu \textit{benchmark} yang digunakan untuk mengukur performa teori kontrol yang digunakan. Sistem ini pada dasarnya terdiri dari beberapa komponen yaitu (i)batang (ii)kereta dan (iii)Motor elektrik dimana batang dari sistem akan dijaga agar berada pada posisi vertikal diatas kereta (\textit{cart}) dengan menggunakan metode kontrol umpan balik. Tulisan ini dibuat sebagai langkah awal untuk mendesain suatu sistem pendulum terbalik dengan metode kontrol PID dengan meninjau persamaan non-linier yang telah diturunkan.}


%----------------------------------------------------------------------------------------

\begin{document}

\flushbottom % Makes all text pages the same height

\maketitle % Print the title and abstract box

\tableofcontents % Print the contents section

\thispagestyle{empty} % Removes page numbering from the first page

%----------------------------------------------------------------------------------------
%	ARTICLE CONTENTS
%----------------------------------------------------------------------------------------

\section{Pendahuluan} %Pendahuluan
\subsection{Latar Belakang}

Sistem pendulum terbalik adalah sistem yang sering digunakan untuk mendemonstrasikan dinamika sistem dan implementasi sistem kontrol \cite{pendh}. Sistem pendulum terbalik dapat digunakan sebagai salah satu \textit{benchmark} pada penerapan teori kontrol. \textit{Benchmark} dari teori kontrol yang dimaksud adalah penggunaan model yang telah diturunkan dan melakukan pengukuran validasi terhadap efisiensi dari metode kontrol yang diimplementasikan pada sistem \cite{bench}.

Sistem pendulum terbalik sendiri adalah sistem non-linear yang terdiri dari suatu batang yang diletakkan pada sebuah kereta (\textit{cart}) yang akan digerakkan menggunakan motor yang akan dikendalikan sehingga mengasilkan gerakan ayunan dari batang yang berfungsi untuk menjaga batang dapat mencapai posisi vertikal diatas kereta (\textit{cart}) \cite{lowcost}

Dalam beberapa tahun terakhir, pengaplikasian dari sistem pendulum terbalik banyak digunakan dalam penelitian \textit{Two-Wheeled Self-Balancing Robot} \cite{twow}\cite{twoww}. Metode kontrol yang digunakan adalah kontrol LQR (\textit{Linear Quadratic Regulator}) karena dibatasi dengan simpangan sudut kecil\cite{twow}. Kekurangan dari penggunaan metode kontrol LQR adalah sudut simpangan yang terlalu kecil karena sistem yang dikendalikan oleh kontrol LQR adalah sistem linear \cite{cdes}.

\subsection{Identifikasi masalah}

Berdasarkan uraian pada latar belakang masalah di atas, dapat diiidentifikasi masalah untuk menyelesaikan penelitian sistem pendulum terbalik adalah:

\begin{enumerate}[noitemsep]
	\item Apa saja komponen dari sistem pendulum terbalik?
	\item Apa saja komponen dari sistem pendulum terbalik yang perlu dimodelkan?
	\item Bagaimana cara memodelkan sistem pendulum terbalik?
	\item Apa Metode kontrol yang tepat untuk mengendalikan motor agar dapat digunakan simpangan maksimal dari sistem pendulum terbalik yaitu $\theta$ = 180\textsuperscript{o}?
\end{enumerate}

\subsection{Batasan masalah}

Agar penelitian Tugas Akhir ini dapat diselesaikan dengan baik, perlu ada batasan pada masalah utama di atas. Batasan masalah tersebut antara lain: 
	
\begin{enumerate}[noitemsep]
	\item Sistem yang akan dibangun masih berupa rancangan purwarupa (prototype).
	\item Sistem pendulum terbalik akan menggunakan satu buah batang.
	\item Sistem akan dibangun dengan basis Arduino atau Raspberry Pi.
\end{enumerate}

\subsection{Tujuan Penelitian}

Tujuan penelitian pada Tugas Akhir ini adalah membuat model dan mengimplementasikan kontrol umpan balik  pada sistem pendulum terbalik dan membuat prototipe dari sistem, serta menampilkan perfroma dari sistem pendulum terbalik menggunakan suatu metode kontroluntuk dapat mempertahankan posisi dari batang yang diinginkan dengan sudut simpangan $\theta$=180\textsuperscript{o}.

\subsection{Manfaat Penelitian}
Berikut adalah manfaat dari penelitian sistem pendulum terbalik untuk beberapa pihak, antara lain: 
\begin{itemize}[noitemsep]
	\item Laboratorium Kontrol yang ingin menunjukkan implementasi sistem kontrol pada suatu sistem (pada kasus ini Sistem Pendulum Terbalik).
	\item Pembaca yang ingin mempelajari pemanfaatan sistem kontrol pada sistem pendulum terbalik.
	\item Peneliti pribadi, untuk menambah pengetahuan dan pengalaman menyelesaikan masalah nyata di lapangan.
	\item Pengembangan ilmu pengetahuan, terutama pada bidang sistem kontrol, dan sistem pengukuran dan akuisisi data.
\end{itemize}

%\addcontentsline{toc}{section}{Pendahuluan} % Adds this section to the table of contents.


%------------------------------------------------

\section{Telaah Pustaka}

\subsection{State Space Control Using LQR Method for a Cart-Inverted Pendulum Linearised Model}
Pada Referensi ini akan diambil bagian penurunan model matematik dari sistem pendulum terbalik. Penulis menggunakan metode euler-lagrange, metode ini menurunkan persamaan dinamika dari sistem dengan menjabarkan energi kinetik dan potensial dari sistem untuk mendapatkan persamaan gerak dari sistem. Illustrasi model sistem pendulum terbalik digambarkan pada gambar \ref{illus}:

\begin{figure}[h]
	\centering
	\includegraphics[width=0.7\linewidth]{illu1}
	\caption{Illustrasi Model Sistem Pendulum Terbalik \cite{ssc}}
	\label{illus}
\end{figure}
\noindent
Persamaan \textit{Euler-Lagrange} dituliskan dengan persamaan \ref{eq:lag1}
\begin{equation}
\label{eq:lag1}
\mathcal{L} = \mathcal{K}(q,\dot{q}) - \mathcal{P}(q)
\end{equation}
dimana 
\begin{itemize}
	\item $\mathcal{L}$ adalah Persamaan Lagrange
	\item $\mathcal{K}(q,\dot{q})$ adalah Energi Kinetik 
	\item $\mathcal{P}(q)$ adalah Energi Potensial
\end{itemize}
\noindent
energi kinetik dituliskan dengan persamaan \ref{eq:ek}
\begin{equation}
\label{eq:ek}
\mathcal{K}(q,\dot{q}) = \frac{1}{2}mv^{2}
\end{equation}
\noindent
energi potensial dituliskan dengan persamaan \ref{eq:ep}
\begin{equation}
\label{eq:ep}
\mathcal{P}(q)= mgh
\end{equation}
dimana
\begin{itemize}
	\item \textit{m} adalah massa benda
	\item \textit{v} adalah turunan posisi
	\item \textit{g} adalah gaya gravitasi
	\item \textit{h} adalah ketinggian benda dari permukaan
\end{itemize}
\noindent
kemudian persamaan gerak dapat diturunkan dari persamaan lagrange yang didapat dengan persamaan \ref{eq:lag2}


\begin{equation}
\label{eq:lag2}
\frac{d}{dt}\frac{\partial\mathcal{L}}{\partial{q_{\dot{i}}}}-\frac{\partial\mathcal{L}}{\partial{q_{i}}} = Q_{i}
\end{equation}
dimana
\begin{itemize}
	\item $Q_{i}$ adalah gaya eksternal yang diberikan pada sistem
	\item $q_{i}$ adalah keluaran dari sistem, i = 1,2,.....
\end{itemize}
\noindent
Persamaan gerak dari sistem pendulum terbalik pada gambar \ref{illus} dituliskan dalam persamaan \ref{eq:eom1} dan \ref{eq:eom2}

\begin{equation}
\label{eq:eom1}
f = (M + m)\ddot{x} + ml\ddot{\theta}cos\theta - ml\dot{\theta^{2}}sin\theta
\end{equation}
dan
\begin{equation}
\begin{split}
\label{eq:eom2}
0 = ml(\ddot{x}cos\theta - \dot{\theta}\dot{x}sin\theta) + ml^{2}\ddot{\theta}\\ -(-m\dot{x}l\dot{\theta}sin\theta + mglsin\theta)
\end{split}
\end{equation}

dimana
\begin{itemize}
	\item \textit{M} adalah massa kereta
	\item \textit{m} adalah massa batang
	\item \textit{$\dot{x}$} adalah turunan pertama dari posisi
	\item \textit{$\ddot{x}$} adalah turunan kedua dari posisi
	\item \textit{$\theta$} adalah sudut kemiringan batang
	\item \textit{$\dot{\theta}$} adalah turunan pertama dari sudut kemiringan batang
	\item \textit{$\dot{\theta}$} adalah turunan kedua dari sudut kemiringan batang
	\item \textit{l} adalah panjang batang	
\end{itemize}

\subsection{Design and development of a low-cost inverted pendulum for control education}
Pada Referensi ini penulis menjelaskan komponen apa saja yang penulis gunakan untuk membuat suatu sistem pendulum terbalik untuk tujuan pendidikan dengan biaya yang rendah, karena sistem pendulum terbalik memiliki karakteristik dinamika yang menarik dan mudah dimengerti bagi orang yang memiliki pengetahuan dasar mengenai mekanika dan pergerakan.
\begin{figure}[h!] % Using \begin{figure*} makes the figure take up the entire width of the page
	\centering
	\includegraphics[width=1\linewidth]{pendas}
	\caption{Rakitan Komponen Sistem Pendulum Terbalik\cite{ssc}}
	\label{fig:pendasli}
\end{figure}

Gambar \ref{fig:pendasli} menggambarkan sebuah sistem pendulum terbalik yang telah dirakit, komponen dari sistem pendulum terbalik dari referensi ini adalah sebagai berikut:

\begin{enumerate}
	\item Batang Pendulum
	\item Kereta \textit{(cart)} 
	\item Motor DC dengan Encoder
	\item Arduino Mega
	\item Limit Switch
	\item Batang Rel untuk Kereta
\end{enumerate}


Motor yang digunakan pada sistem pendulum terbalik ini adalah \textit{motor stepper}, motor ini digunakan untuk mendapatkan sudut dari putaran motor sesuai yang diinginkan oleh pengguna. Pada referensi ini tidak dijelaskan mengapa motor stepper yang dipilih adalah motor dengan tipe Trinamic QSH618-45-28-110.

Sistem pendulum terbalik memerlukan sensor posisi dan sudut untuk mengetahui posisi dari kereta \textit{(cart)} dan batang pendulum, maka dari itu diperlukan \textit{encoder} dari untuk mengetahui sudut dari \textit{shaft} motor dan kecepatan dari gerakan kereta \textit{(cart)}. 

Untuk keamanan sistem agar mencegah sistem berfungsi secara abnormal, maka digunakan dua buah sensor \textit{limit switch} yang berfungsi untuk memutus sumber listrik ke motor agar kereta tidak bergerak melewati batas yang telah ditentukan yang telah diprogram pada mikrokontroller Arduino Mega. 

\subsection{Analysis \& Control of Inverted Pendulum System Using PID Controller}
Pada Referensi ini penulis menggunakan pengendali PID, namun untuk melakukan penyetelan nilai parameter padapengendali PID umumnya dilakukan menggunakan metode \textit{trial \& error}. Penulis menyatakan pengendali LQR itu setara dengan pengedali PD dua \textit{loop} yang memiliki respon yang buruk, maka dari itu digunakanlah pengendali PID dan penyetelan dari nilai parameter PID menggunakan metode \textit{pole-placement}.

Terdapat dua pengendali PID pada sistem yang dibuat oleh penulis, pengendali PID digunakan untuk mengendalikan posisi dari kereta \textit{(cart)} dan sudut kemiringan dari batang pendulum. Persamaan \ref{eq:pid} adalah persamaan umum dari pengendali PID
\begin{equation}
\label{eq:pid}
u(t) = K_{p}e(t) + K_{i}\int_{0}^{t}e(t)dt + K_{d}\frac{de(t)}{dt}
\end{equation}
dimana:

\begin{itemize}
	\item u(t) adalah gaya eksternal sistem
	\item $e(t)$ adalah nilai perbedaan \textit{feedback} dengan \textit{reference}
	\item $K_{p}$ adalah \textit{gain} proporsional
	\item $K_{i}$ adalah \textit{gain} integratif
	\item $K_{d}$ adalah \textit{gain} derivatif
\end{itemize}

Pengendali PID pertama akan digunakan untuk mengendalikan posisi sedangkan PID kedua akan digunakan untuk mengendalikan sudut $\theta$. maka PID satu adalah PID1 = X(s)/U(s) dan PID dua adalah PID2 = $\theta$(s)/U(S), persamaan ini didapat dari persamaan sistem secara umum yaitu \textit{output/input} = sistem. 

Persamaan karakteristik kontrol didapat dengan persamaan \ref{eq:belt}
\begin{equation}
\label{eq:belt}
1 - PID1C1 + PID2C2 = 0
\end{equation}
\noindent
dimana C1 dan C2 adalah persamaan umum dari pengendali PID. maka didapatkan lah persamaan karakteristik yang diinginkan pada persamaan \ref{eq:belt2}

\begin{equation}
\label{eq:belt2}
{s^5} + {p1s^4} + {p2s^3} + {p3s^2} + p4s + p5
\end{equation}

\begin{figure}[h!]
	\centering
	\includegraphics[width=1\linewidth]{db}
	\caption{Diagram Blok PID dua loop \cite{PIDC}}
	\label{dbo}
\end{figure}

%------------------------------------------------

\section{Hasil dan Pembahasan}
Berikut adalah Hasil yand didapatkan setelah menelaah tiga buah referensi penelitian mengenai sistem pendulum terbalik

\begin{enumerate}
\item Terdapat dua cara untuk menurunkan model dinamika suatu sistem, menggunakan dinamika lagrange adalah salah satu caranya. model dinamika dari sistem pendulum terbalik dapat diliahat pada persamaan \ref{eq:eom1} dan \ref{eq:eom2} dengan menggunakan komponen sesuai pada referensi \cite{cdes}
\item Hasil yang didapatkan penulis pada referensi \cite{PIDC} menyatakan bahwa dengan menggunakan pengendali PID dengan penyetelan parameter nilai PID dengan menggunakan metode \textit{pole placement} menghasilkan hasil respon yang stabil dan memiliki \textit{rise time} dan \textit{overshoot} yang rendah dari posisi dan sudut yang dihasilkan oleh kereta \textit{cart} dan batang pendulum yang digambarkan pada gambar \ref{suduts} dan \ref{posisis}

\end{enumerate}
\begin{figure}[h!]
	\centering
	\includegraphics[width=1\linewidth]{sudut}
	\caption{Hasil grafik sudut batang \cite{PIDC}}
	\label{suduts}
\end{figure}
\begin{figure}[h!]
	\centering
	\includegraphics[width=1\linewidth]{posisi}
	\caption{Hasil grafik posisi kereta \textit{cart} \cite{PIDC}}
	\label{posisis}
\end{figure}



\section{Kesimpulan}

Berikut adalah Kesimpulan dari Telaah Pustaka dari beberapa referensi yang telah dilakukan:
\textit{\begin{enumerate}
	\item Telah membaca \cite{cdes} dan \cite{ssc} dan telah memahami Model Dinamika Lagrange.
	\item Telah menurunkan model lagrange dari sistem pendulum terbalik.
	\item Telah mengetahui komponen dasar yang diperlukan untuk membuat sistem pendulum terbalik.
	\item akan melanjutkan membaca \cite{ssc} dan \cite{PIDC}
	\item Akan melanjutkan membaca tentang Model Dinamika suatu sistem.
	\item Akan mencari referensi memodelkan motor elektrik.
	\item Akan mencari referensi kontrol PID lainnya
\end{enumerate}}



%--------------------------------------------------------------------------------------------DAFTAR-PUSTAKA----------------------------------------------------------------------------------------

\phantomsection
\begin{thebibliography}{99}
\bibitem{pendh} Kent H. Lundberg and Taylor W. Barton. \newblock {\em History of Inverted-Pendulum Systems.} International Federation of Automatic Control, 2010.
\bibitem{bench} Olfa Boubaker. \newblock {\em The Inverted Pendulum: A fundamental Benchmark in Control Theory and Robotics}. \textit{In Education and e-LEarning Innovations}, Self Published, 2012.
\bibitem{twow}Hellman, Hanna and Sunnerman, Henrik. \textit{Two-Wheeled Self-Balancing Robot: Design and control based on the concept of an inverted pendulum}. London: KTH, 2015.
\bibitem{twoww}Ooi, Rich Chi \textit{Balancing a Two-Wheeled Autonomous Robot}. Perth: The University of Western Australia, 2003.
\bibitem{cdes}P. Kumar and J.Mahto \textit{CONTROLLER DESIGN OF INVERTED PENDULUM USING POLE PLACEMENT AND LQR}. IJRET, 2012.
\bibitem{lowcost}Bakar{\'a}{\v{c}}, Peter and Kal{\'u}z, Martin and others \textit{Design and development of a low-cost inverted pendulum for control education}. IEEE, 2017.
\bibitem{ssc} Indrazno Siradjuddin, Budhy Setiawan, And Ahmad Fahmi \textit{State Space Control Using LQR Method for a Cart-Inverted Pendulum Linearised Model}. IJMME, 2017.
\bibitem{PIDC} Vivek Kumar pathak and Sankalp Paliwal
\textit{Analysis \& Control of Inverted Pendulum System Using PID}.IJERA, 2017.
\end{thebibliography}


%----------------------------------------------------------------------------------------

\end{document}